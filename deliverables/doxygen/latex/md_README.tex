\subsection*{Members\+:}


\begin{DoxyItemize}
\item Alan Gomez
\item Samuel Parra
\item Brennan Penfold
\end{DoxyItemize}

\subsection*{Client}


\begin{DoxyItemize}
\item Djordje Vukcevic
\end{DoxyItemize}

\subsection*{Description}

This repository is a group submission for M\+AS Software Development Project. The goal of the project is to implement a given software algorithm from a research paper, while communicating with a client to add desired features. The paper implemented is a method of obstacle avoidance for robotic manipulators and at the request of the client a combination of 2 papers was implemented and can be found in the reference section \mbox{[}1\mbox{]}\mbox{[}2\mbox{]}

\subsection*{Structure}

This repository is structured to contain multiple implementations/libraries and documents for the final submission. The repository has a hierarchial structure based around the Collision Monitoring library leading to the final R\+OS package used for demonstration. The following links list the core functionalities of this repository.

\subsubsection*{Contents}


\begin{DoxyItemize}
\item \hyperlink{collision__monitoring_2_r_e_a_d_m_e_8md}{Collision Monitoring Library}\+: The base library designed to be highly portable and used for any manipulator type
\item \hyperlink{md_src_README}{Kinova Arm Package}\+: The test implementation of the Collision monitoring library designed for a Kinova gen3 7\+D\+OF manipulator.
\item \hyperlink{md_catkin_workspace_README}{Collision Avoidance R\+OS Package}\+: The implementation of the obstacle avoidance and tracing algorithm that the base library is designed to suit.
\item Documentation \& other related works\+: The last section found in the /deliverables directory contains the paper, presentation and A\+PI documentation which is submitted along with this code.
\end{DoxyItemize}

\subsubsection*{Folder layout}

Here is the basic layout of the folder structure (note\+: symbolic links are used to for easy modification and tracking)\+:


\begin{DoxyCode}
1 Repository
2 │
3 └─── build/                    Where the KinovaArm package is to be built
4 │
5 └─── test/tests                The test file based off catch
6 |
7 └─── catkin\_workspace/         The ros catkin workspace (run catkin\_make here)
8 |
9 └─── docs/                     Contains UML diagrams associated with the ROS pkg
10 └─── src/kinova\_arm/           The ROS package used for testing and demonstration
11 |
12 └─── collision\_monitoring/     The base library built for the project
13 |   |
14 |   └─── docs/                 Contains associated UML diagrams
15 |   └─── include/              The library header files
16 |   └─── src/                  The libraries source files
17 |
18 └─── deliverables/             Contains other non-code submittable documents
19 |   |
20 |   └─── doxygen/              Contains the html API docs and files to make them
21 |   └─── latex/                Contains the latex files for the research paper
22 |   └─── presentation/         Contains the final presentation
23 |
24 └─── docs/                     Contains the UML files for the kinova\_arm package
25 └─── include/                  Contains the header files for kinova\_arm and ROS
26 └─── src/                      Contains the source files for kinova\_arm and ROS
27 └─── test/                     Contains the test source files for kinova\_arm
28 └─── urdf/                     Contains the URDF file used for kinematic calcs
\end{DoxyCode}


\subsection*{License}

This project is licensed under the M\+IT License -\/ see the \hyperlink{md_LICENSE}{L\+I\+C\+E\+N\+SE.md} file for details

\subsection*{Acknowledgments}


\begin{DoxyItemize}
\item We would like to thank our coach Djordje Vukcevic who supported us throughout the development of this library.
\end{DoxyItemize}

\subsection*{References}

This library is an implementation of the following papers\+:


\begin{DoxyItemize}
\item \mbox{[}1\mbox{]} Real-\/\+Time Obstacle Avoidance for Manipulators and Mobile Robots, Oussama Khatib, 1986.
\item \mbox{[}2\mbox{]} Biologically-\/inspired dynamical systems for movement generation\+: automatic real-\/time goal adaptation and obstacle avoidance, H. Hoffmann et al., 2019. 
\end{DoxyItemize}