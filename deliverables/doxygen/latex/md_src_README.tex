The package that implements the Collision monitoring library for the Kinova gen3 7\+D\+OF arm. This not only is done for testing but also gives an example of how the \hyperlink{class_arm}{Arm} interface can be implemented.

\subsection*{Documentation}

For U\+ML diagrams have a look in the /docs folder. A html A\+PI style documentation can be found under /deliverables/doxygen/html

\subsection*{Requirements\+:}


\begin{DoxyItemize}
\item K\+DL (\href{https://www.orocos.org/kdl/installation-manual}{\tt https\+://www.\+orocos.\+org/kdl/installation-\/manual})
\item Eigen (\href{http://eigen.tuxfamily.org}{\tt http\+://eigen.\+tuxfamily.\+org})
\item C\+M\+A\+KE V3.\+5
\end{DoxyItemize}

\subsection*{Building}

In the root folder of the repository create a build folder\+: 
\begin{DoxyCode}
1 ...$ mkdir build
\end{DoxyCode}
 cd into the build folder and run cmake for the first time to configure the build environment\+: 
\begin{DoxyCode}
1 ...$ cd build
2 ...\(\backslash\)build$ cmake ..
\end{DoxyCode}
 If VS Code is being used the C\+M\+A\+KE Tools extension is helpful as it allows the library to be built by pressing F7.

Otherwise the following commands need to be run from the build folder\+: 
\begin{DoxyCode}
1 ...\(\backslash\)build$ cmake ..
2 ...\(\backslash\)build$ cmake --build .
\end{DoxyCode}


\subsection*{Testing}

The test files are built in build/tests and need to be run from the build directory because of a relative file reference. These can be run using the following commands\+: 
\begin{DoxyCode}
1 ...\(\backslash\)build$ ./test/tests
\end{DoxyCode}


\subsection*{Notes on files\+:}

The only following files are used in kinova\+\_\+arm and other files present in the /src and /include directories are for the R\+OS implementation. For more information on the R\+OS implementation see the R\+O\+S\+\_\+\+Package\+\_\+\+R\+E\+A\+D\+M\+E.\+md.


\begin{DoxyItemize}
\item \hyperlink{kinova__arm_8cpp}{kinova\+\_\+arm.\+cpp}
\item \hyperlink{kinova__arm_8h}{kinova\+\_\+arm.\+h}
\end{DoxyItemize}

\subsection*{U\+ML diagram}



 