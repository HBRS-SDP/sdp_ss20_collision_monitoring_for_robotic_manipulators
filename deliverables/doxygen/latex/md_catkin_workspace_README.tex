The package that implements the Kinova \hyperlink{class_arm}{Arm} package in a ros package for demonstration purposes.

\subsection*{Documentation}

For U\+ML diagrams have a look in the /docs folder and for basic R\+OS commands see \href{http://wiki.ros.org/ROS/Tutorials}{\tt http\+://wiki.\+ros.\+org/\+R\+O\+S/\+Tutorials}.

\subsection*{Requirements\+:}


\begin{DoxyItemize}
\item K\+DL (\href{https://www.orocos.org/kdl/installation-manual}{\tt https\+://www.\+orocos.\+org/kdl/installation-\/manual})
\item Eigen (\href{http://eigen.tuxfamily.org}{\tt http\+://eigen.\+tuxfamily.\+org})
\item R\+OS kinetic (\href{http://wiki.ros.org/}{\tt http\+://wiki.\+ros.\+org/})
\item C\+M\+A\+KE V3.\+5
\end{DoxyItemize}

\subsection*{Packaged dependencies}

The kinova kortex library is packaged with this catkin workspace for easy building and implementation. This is pulled directly from the \href{https://github.com/kinovarobotics/ros_kortex}{\tt ros kortex} github page and the associated licences and readme can be found in the catkin/src directory.

\subsection*{Building}

In the catkin\+\_\+workspace folder of the repository make sure you have ros setup.\+bash sourced in the current terminal\+: 
\begin{DoxyCode}
1 ...\(\backslash\)catkin\_workspace$ source /opt/ros/kinetic/setup.bash 
\end{DoxyCode}
 Then run catkin build (or catkin\+\_\+make\+: 
\begin{DoxyCode}
1 ...\(\backslash\)catkin\_workspace$ catkin build
\end{DoxyCode}
 Finally source the new setup.\+bash, or add it to your .bashrc to be run on the terminals startup\+: 
\begin{DoxyCode}
1 ...\(\backslash\)catkin\_workspace$ source devel/setup.bash
\end{DoxyCode}


\subsection*{Running the software}

There are 2 modes the software can be run in, single or dual arm, each with its separate interface node types.

\subsubsection*{Single arm}

First make sure that the correct setup.\+bash file is sourced. 
\begin{DoxyCode}
1 ...\(\backslash\)catkin\_workspace$ source devel/setup.bash
\end{DoxyCode}
 Then launch the single arm simulation with\+: 
\begin{DoxyCode}
1 ...$ roslaunch kinova\_arm single\_manipulator.launch 
\end{DoxyCode}
 This will bring up the rviz display where the kinova arm model can be seen. To interface with the model the single interfacer node needs to be opened, so in a new terminal source the setup file (if you didn\textquotesingle{}t add it to your .bashrc) and then run\+: 
\begin{DoxyCode}
1 ...$ rosrun kinova\_arm KinovaInterface 
\end{DoxyCode}
 Where using basic keyboard commands you can set goals and add obstacles to the environment.

\subsubsection*{Dual arm}

First make sure that the correct setup.\+bash file is sourced. 
\begin{DoxyCode}
1 ...\(\backslash\)catkin\_workspace$ source devel/setup.bash
\end{DoxyCode}
 Then launch the dual arm simulation with\+: 
\begin{DoxyCode}
1 ...$ roslaunch kinova\_arm dual\_manipulators.launch  
\end{DoxyCode}
 This will bring up the rviz display where the kinova arm models can be seen. To interface with the model the dual interfacer node needs to be opened, so in a new terminal source the setup file (if you didn\textquotesingle{}t add it to your .bashrc) and then run\+: 
\begin{DoxyCode}
1 ...$ rosrun kinova\_arm DualArmKinovaInterface 
\end{DoxyCode}
 Where using basic keyboard commands you can set goals and add obstacles to the environment. 